\section{Introducción}\label{introducciuxf3n}

\begin{frame}{Objetivos de la sesión}
\phantomsection\label{objetivos-de-la-sesiuxf3n}
\begin{itemize}
\tightlist
\item
  Entender qué significa que un modelo \emph{aprenda}
\item
  Diferenciar generalización, sobreajuste y subajuste
\item
  Reconocer los principales componentes de un proyecto de ML
\item
  Ubicar arquitecturas y optimizadores en el panorama general
\end{itemize}
\end{frame}

\begin{frame}
\end{frame}

\section{¿Qué significa aprender?}\label{quuxe9-significa-aprender}

\begin{frame}{Idea central}
\phantomsection\label{idea-central}
Un modelo aprende cuando:\\
- captura \textbf{patrones reales}\\
- funciona bien en \textbf{datos no vistos}\\
Aprender \textbf{no es memorizar}.
\end{frame}

\begin{frame}{}
\phantomsection\label{section}
\begin{quote}
¿Un modelo que tiene 100 \% de accuracy en entrenamiento necesariamente
aprendió?
\end{quote}
\end{frame}

\begin{frame}
\end{frame}

\section{Generalización}\label{generalizaciuxf3n}

\begin{frame}{Concepto clave}
\phantomsection\label{concepto-clave}
\textbf{Generalizar} significa que el modelo: - funciona bien en datos
nuevos - no depende de ejemplos específicos

La generalización es el objetivo real del entrenamiento.
\end{frame}

\begin{frame}{}
\phantomsection\label{section-1}
\begin{quote}
¿Por qué nos importa más el rendimiento en validación que en
entrenamiento?
\end{quote}
\end{frame}

\begin{frame}
\end{frame}

\section{Sobreajuste}\label{sobreajuste}

\begin{frame}{Overfitting}
\phantomsection\label{overfitting}
Ocurre cuando el modelo: - aprende detalles irrelevantes (ruido) -
pierde capacidad de generalizar

\textbf{Señales típicas:} - pérdida muy baja en entrenamiento - pérdida
alta en validación
\end{frame}

\begin{frame}{}
\phantomsection\label{section-2}
\begin{quote}
¿Qué tipo de modelo es más propenso al sobreajuste: uno simple o uno muy
complejo?
\end{quote}
\end{frame}

\begin{frame}
\end{frame}

\section{Subajuste}\label{subajuste}

\begin{frame}{Underfitting}
\phantomsection\label{underfitting}
Ocurre cuando el modelo: - no tiene suficiente capacidad - no captura el
patrón real

\textbf{Señales típicas:} - pérdida alta en entrenamiento - pérdida alta
en validación
\end{frame}

\begin{frame}{}
\phantomsection\label{section-3}
\begin{quote}
¿Puede un modelo subajustado mejorar solo entrenando más tiempo?
\end{quote}
\end{frame}

\begin{frame}
\end{frame}

\section{Técnicas para mejorar la
generalización}\label{tuxe9cnicas-para-mejorar-la-generalizaciuxf3n}

\begin{frame}{Visión general}
\phantomsection\label{visiuxf3n-general}
Algunas estrategias comunes: - Regularización - Dropout - Early stopping
- Más y mejores datos

Todas buscan el mismo objetivo: \textbf{controlar la complejidad
efectiva del modelo}.
\end{frame}

\begin{frame}{}
\phantomsection\label{section-4}
\begin{quote}
¿Estas técnicas cambian el modelo o cambian cómo se entrena?
\end{quote}
\end{frame}

\begin{frame}
\end{frame}

\section{Importancia de los datos}\label{importancia-de-los-datos}

\begin{frame}{Antes del modelo}
\phantomsection\label{antes-del-modelo}
\begin{itemize}
\tightlist
\item
  Calidad \textgreater{} cantidad
\item
  Datos mal preparados → modelos pobres
\item
  El modelo solo puede aprender lo que los datos contienen
\end{itemize}

Procesos comunes: - normalización - estandarización - balanceo
\end{frame}

\begin{frame}{}
\phantomsection\label{section-5}
\begin{quote}
¿Un modelo muy avanzado puede compensar datos de mala calidad?
\end{quote}
\end{frame}

\begin{frame}
\end{frame}

\section{Flujo general de un proyecto de
ML}\label{flujo-general-de-un-proyecto-de-ml}

\begin{frame}{Panorama completo}
\phantomsection\label{panorama-completo}
\begin{enumerate}
\tightlist
\item
  Recolección de datos\\
\item
  Limpieza y procesamiento\\
\item
  División entrenamiento / validación\\
\item
  Selección del modelo\\
\item
  Entrenamiento\\
\item
  Evaluación\\
\item
  Ajuste y regularización\\
\item
  Implementación
\end{enumerate}

Este flujo se repite iterativamente.
\end{frame}

\begin{frame}{}
\phantomsection\label{section-6}
\begin{quote}
¿En qué etapa crees que se cometen más errores en la práctica?
\end{quote}
\end{frame}

\begin{frame}
\end{frame}

\section{Arquitecturas modernas}\label{arquitecturas-modernas}

\begin{frame}{Qué tipos de modelos existen}
\phantomsection\label{quuxe9-tipos-de-modelos-existen}
\begin{itemize}
\tightlist
\item
  \textbf{Redes densas}: datos tabulares
\item
  \textbf{CNNs}: imágenes
\item
  \textbf{RNNs / LSTM}: secuencias
\item
  \textbf{Transformers}: texto, visión, multimodal
\end{itemize}

Hoy veremos \emph{qué existen}, no \emph{cómo funcionan}.
\end{frame}

\begin{frame}{}
\phantomsection\label{section-7}
\begin{quote}
¿Por qué no usamos la misma arquitectura para todos los problemas?
\end{quote}
\end{frame}

\begin{frame}
\end{frame}

\section{Optimizadores modernos}\label{optimizadores-modernos}

\begin{frame}{Visión general (sin entrar en detalles)}
\phantomsection\label{visiuxf3n-general-sin-entrar-en-detalles}
Durante el entrenamiento ajustamos parámetros usando optimizadores:

\begin{itemize}
\tightlist
\item
  \textbf{SGD}: descenso básico
\item
  \textbf{Momentum}: acumula dirección
\item
  \textbf{RMSProp}: adapta el paso por parámetro
\item
  \textbf{Adam}: combina velocidad y estabilidad
\end{itemize}

Los estudiaremos \textbf{en profundidad más adelante}.
\end{frame}

\begin{frame}{}
\phantomsection\label{section-8}
\begin{quote}
¿Por qué tendría sentido usar distintos optimizadores para el mismo
modelo?
\end{quote}
\end{frame}

\begin{frame}
\end{frame}

\section{Idea clave de la sesión}\label{idea-clave-de-la-sesiuxf3n}

\begin{frame}{Todo se conecta}
\phantomsection\label{todo-se-conecta}
\begin{itemize}
\tightlist
\item
  Datos influyen en generalización
\item
  Arquitectura define capacidad
\item
  Optimizadores controlan el aprendizaje
\item
  El objetivo final es \textbf{generalizar}
\end{itemize}

Hoy construimos el mapa; luego veremos el mecanismo.
\end{frame}

\begin{frame}{}
\phantomsection\label{section-9}
\begin{quote}
¿Qué parte del proceso te genera más curiosidad o incertidumbre ahora?
\end{quote}
\end{frame}

\begin{frame}
\end{frame}

\section{Próxima sesión}\label{pruxf3xima-sesiuxf3n}

\begin{frame}{Cómo ocurre el aprendizaje realmente}
\phantomsection\label{cuxf3mo-ocurre-el-aprendizaje-realmente}
En la próxima sesión veremos: - tensores - gradientes - grafos
computacionales - cómo PyTorch calcula derivadas

Ahí empieza el ``cómo''.
\end{frame}

\begin{frame}{}
\phantomsection\label{section-10}
\begin{quote}
¿Qué crees que hace posible entrenar redes con millones de parámetros?
\end{quote}
\end{frame}
